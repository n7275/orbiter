\documentclass[Orbiter Technical Reference.tex]{subfiles}
\begin{document}

\section{Dynamic vessel-surface interaction}
\textbf{Martin Schweiger}


\subsection{Introduction}
%TODO introduction
blah

\subsection{Equilibrium landing gear compression}\label{ssec:equi_comp}
For seamless transition between dynamic and deterministic state updates for vessels resting on a planetary surface, we must be able to calculate the compression of landing gear at equilibrium, given the magnitude of gravitational acceleration $g$ on the surface, the vessel mass $m$, and the positions $\vec{r}_i = \lbrace x_i, y_i, z_i\rbrace$ and suspension constants $\alpha_i$ of the individual landing gear components (or equivalent) $i$. The compression has to be included in the touchdown position calculation for the deterministic model, to avoid vertical jumps when switching between update models. Orbiter currently uses 3 touchdown points to define a vessel's surface interaction, but the approach is valid for an arbitrary number of points.

First, assume that the plane described by the three touchdown points is parallel to the vessel's $y=0$ plane, and that any tilt of the vessel at rest induced by different loads on the gear elements is small. Then the compression effect on each gear is represented by a change $\Delta y_i$ of the gear's $y$-coordinate.

To calculate $\Delta y_i$, $i=1,2,3$, we first state the equilibrium conditions for the linear and angular moments around $x$ and $z$ axes:
\begin{equation}\label{eq:equi}
\begin{split}
\sum F &= \sum_{i=1}^{n} \alpha_i \Delta y_i - m g = 0 \\
\sum M_x &= \sum_{i=1}^{n} x_i \alpha_i \Delta y_i = 0 \\
\sum M_z &= \sum_{i=1}^{n} z_i \alpha_i \Delta y_i = 0
\end{split}
\end{equation}
The solution is constrained by the fact that the vessel is considered a rigid body. This is exploited by expressing the solution in terms of the $y$-displacement of the centre of gravity, $\Delta Y$, and two rotations $\sin \theta$ and $\sin \phi$ around the x and z axes, respectively. Then, assuming the x and z tilts are small enough to decouple, the landing gear compressions are given in terms of the unknowns by
\begin{equation}\label{eq:prm}
\Delta y_i = \Delta Y + \sin \theta x_i + \sin \phi z_i
\end{equation}
Subsituting this into Equations~\ref{eq:equi} leads to
\begin{equation}
\begin{split}
&\Delta Y \sum_{i=1}^{n} \alpha_i + \sin\theta \sum_{i=1}^{n} \alpha_i x_i + \sin\phi \sum_{i=1}^{n} \alpha_i z_i = mg \\
&\Delta Y \sum_{i=1}^{n} \alpha_i x_i + \sin\theta \sum_{i=1}^{n} \alpha_i x_i^2 + \sin\phi \sum_{i=1}^{n} \alpha_i x_i z_i = 0 \\
&\Delta Y \sum_{i=1}^{n} \alpha_i z_i + \sin\theta \sum_{i=1}^{n} \alpha_i x_i z_i + \sin\phi \sum_{i=1}^{n} \alpha_i z_i^2 = 0
\end{split}
\end{equation}
so we need to solve the symmetric $3\times 3$ linear system
\begin{equation}\label{eq:linsys}
\left[\begin{array}{ccc}
a&b&c \\
b&d&e \\
c&e&f \end{array}\right]
\left[\begin{array}{c}
\Delta Y \\ \sin\theta \\ \sin\phi \end{array}\right]
=
\left[\begin{array}{c}
mg \\ 0 \\ 0 \end{array}\right]
\end{equation}
with
\begin{equation}
\begin{array}{lll}
a = \sum \alpha_i, & b = \sum \alpha_i x_i, & c = \sum \alpha_i z_i, \\
d = \sum \alpha_i x_i^2, & e = \sum \alpha_i x_i z_i, & f = \sum \alpha_i z_i^2
\end{array}
\end{equation}

If the plane described by the touchdown point positions is not parallel to the $y=0$ plane, then without loss of generality we first perform a rotation of the points to satisfy the condition, apply the compression calculation, and rotate the results back.
\begin{equation}
\vec{r}^{(c)} = R^{-1} C [R \vec{r}]
\end{equation}
where $R$ is the required rotation, and $C$ denotes the operation of applying Equations \ref{eq:linsys} and \ref{eq:prm} on the rotated points.

\subsection{Acknowlegements}
I want to thank RisingFury and RAF92\_Moser, and in particular Miner1 of orbiter-forum.com for their input to the equilibrium landing gear compression problem. The solution in Section~\ref{ssec:equi_comp} follows Miner1's 2D solution. Any mistakes in the 3D extension are mine.

\end{document}
